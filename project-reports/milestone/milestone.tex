%----------------------------------------------------------------------------------------
%	PACKAGES AND OTHER DOCUMENT CONFIGURATIONS
%----------------------------------------------------------------------------------------

\documentclass[paper=a4, fontsize=11pt]{scrartcl} % A4 paper and 11pt font size

\usepackage[T1]{fontenc} % Use 8-bit encoding that has 256 glyphs
%\usepackage{fourier} % Use the Adobe Utopia font for the document - comment this line to return to the LaTeX default
\usepackage[english]{babel} % English language/hyphenation
\usepackage{amsmath,amsfonts,amsthm} % Math packages
\usepackage{graphicx}
\usepackage{listings}
\usepackage{color}

\definecolor{dkgreen}{rgb}{0,0.6,0}
\definecolor{gray}{rgb}{0.5,0.5,0.5}
\definecolor{mauve}{rgb}{0.58,0,0.82}

\lstset{frame=tb,
  language=python,
  aboveskip=3mm,
  belowskip=3mm,
  showstringspaces=false,
  columns=flexible,
  basicstyle={\small\ttfamily},
  numbers=none,
  numberstyle=\tiny\color{gray},
  keywordstyle=\color{blue},
  commentstyle=\color{dkgreen},
  stringstyle=\color{mauve},
  breaklines=true,
  breakatwhitespace=true,
  tabsize=3
}
%\usepackage{lipsum} % Used for inserting dummy 'Lorem ipsum' text into the template

\usepackage{sectsty} % Allows customizing section commands
%\allsectionsfont{\centering \normalfont\scshape} % Make all sections centered, the default font and small caps

\usepackage{fancyhdr} % Custom headers and footers
\pagestyle{fancyplain} % Makes all pages in the document conform to the custom headers and footers
\fancyhead{} % No page header - if you want one, create it in the same way as the footers below
\fancyfoot[L]{} % Empty left footer
\fancyfoot[C]{} % Empty center footer
\fancyfoot[R]{\thepage} % Page numbering for right footer
\renewcommand{\headrulewidth}{0pt} % Remove header underlines
\renewcommand{\footrulewidth}{0pt} % Remove footer underlines
\setlength{\headheight}{13.6pt} % Customize the height of the header

\numberwithin{equation}{section} % Number equations within sections (i.e. 1.1, 1.2, 2.1, 2.2 instead of 1, 2, 3, 4)
\numberwithin{figure}{section} % Number figures within sections (i.e. 1.1, 1.2, 2.1, 2.2 instead of 1, 2, 3, 4)
\numberwithin{table}{section} % Number tables within sections (i.e. 1.1, 1.2, 2.1, 2.2 instead of 1, 2, 3, 4)

%\setlength\parindent{0pt} % Removes all indentation from paragraphs - comment this line for an assignment with lots of text

%----------------------------------------------------------------------------------------
%	TITLE SECTION
%----------------------------------------------------------------------------------------

\newcommand{\horrule}[1]{\rule{\linewidth}{#1}} % Create horizontal rule command with 1 argument of height

\title{	
\normalfont \normalsize 
\textsc{Stanford CS 229 Fall 2017} \\ [25pt] % Your university, school and/or department name(s)
\horrule{0.5pt} \\[0.4cm] % Thin top horizontal rule
\huge Final Project Milestone\\ % The assignment title
\horrule{2pt} \\[0.5cm] % Thick bottom horizontal rule
}

\author{Yang "Eddie" Chen, Yi Zhong, Yubo Tian} % Your name

\date{\normalsize\today} % Today's date or a custom date

\begin{document}

\maketitle % Print the title

%----------------------------------------------------------------------------------------
%	Section 1 Motivation
%----------------------------------------------------------------------------------------

\section{Motivation}
Tennis matches are fun to watch because they are full of surprises. In the 2017 Stuttgart Open, Roger Federer, then an 18-time grand slam champion, lost to the world No. 302 player Tommy Haas. Federer lost the opening match at a grass-court tournament, a phenomenon that hadn't happened since 2002. How can we predict a rare loss like this? Based on past performance alone, any model would have predicted a Federer win in pre-game bets. This motivates us to apply machine learning to predict tennis matches in real time; in particular, we want to explore how well we can combine both historical performance data and real-time, just-happened set-by-set outcome, to achieve a better prediction. As a consequence, both tennis players and sports bettors can benefit from better predictions and new insights. \\

Tennis is an ideal candidate for a hierarchical model as a match consists of a sequence of sets, which in turn consist of a sequence of games, which in turn consist of a sequence of points...\textbf{TODO}: add more here to explain we start by looking at set-by-set\\

This paper seeks to model men's professional singles matches...\textbf{TODO}: \textit{add some introduction after we have better ideas} \\

\textbf{TODO}: add literature review
%----------------------------------------------------------------------------------------
%	Section 2 - Method
%----------------------------------------------------------------------------------------

\section{Method}

%----------------------------------------------------------------------------------------

%----------------------------------------------------------------------------------------
%	Section Preliminary Experiments
%----------------------------------------------------------------------------------------

\section{Preliminary Experiments}


%----------------------------------------------------------------------------------------
%	Section Next Steps
%----------------------------------------------------------------------------------------

\section{Next Steps}

%----------------------------------------------------------------------------------------
%	Section Contributions
%----------------------------------------------------------------------------------------

\section{Contributions}
\subsection{Yi Zhong}
\subsection{Yubo Tian}
\subsection{Yang "Eddie" Chen}

%----------------------------------------------------------------------------------------
%	Conclusion
%----------------------------------------------------------------------------------------

\section{Conclusion}

\appendix
\section{Appendix: Roadmap with Extensions}
\begin{itemize}
\item \textbf{11/20 - 11/26}: 
\item \textbf{11/27 - 12/03}: 
\item \textbf{12/04 - 12/10}: Poster making
\item \textbf{12/11 - 12/15}: Poster presentation; final project write-up
\end{itemize}

\end{document}